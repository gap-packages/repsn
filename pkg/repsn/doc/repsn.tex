% generated by GAPDoc2LaTeX from XML source (Frank Luebeck)
\documentclass[11pt]{report}
\usepackage{a4wide}
\sloppy
\pagestyle{myheadings}
\usepackage{amssymb}
\usepackage[latin1]{inputenc}
\usepackage{makeidx}
\makeindex
\usepackage{color}
\definecolor{DarkOlive}{rgb}{0.1047,0.2412,0.0064}
\definecolor{FireBrick}{rgb}{0.5812,0.0074,0.0083}
\definecolor{RoyalBlue}{rgb}{0.0236,0.0894,0.6179}
\definecolor{RoyalGreen}{rgb}{0.0236,0.6179,0.0894}
\definecolor{RoyalRed}{rgb}{0.6179,0.0236,0.0894}
\definecolor{LightBlue}{rgb}{0.8544,0.9511,1.0000}
\definecolor{Black}{rgb}{0.0,0.0,0.0}
\definecolor{FuncColor}{rgb}{1.0,0.0,0.0}
%% strange name because of pdflatex bug:
\definecolor{Chapter }{rgb}{0.0,0.0,1.0}

\usepackage{fancyvrb}

\usepackage{pslatex}

\usepackage[
        a4paper=true,bookmarks=false,pdftitle={Written with GAPDoc},
        pdfcreator={LaTeX with hyperref package / GAPDoc},
        colorlinks=true,backref=page,breaklinks=true,linkcolor=RoyalBlue,
        citecolor=RoyalGreen,filecolor=RoyalRed,
        urlcolor=RoyalRed,pagecolor=RoyalBlue]{hyperref}

% write page numbers to a .pnr log file for online help
\newwrite\pagenrlog
\immediate\openout\pagenrlog =\jobname.pnr
\immediate\write\pagenrlog{PAGENRS := [}
\newcommand{\logpage}[1]{\protect\write\pagenrlog{#1, \thepage,}}
\newcommand{\Q}{\mathbb{Q}}
\newcommand{\R}{\mathbb{R}}
\newcommand{\C}{\mathbb{C}}
\newcommand{\Z}{\mathbb{Z}}
\newcommand{\N}{\mathbb{N}}
\newcommand{\F}{\mathbb{F}}

\newcommand{\GAP}{\textsf{GAP}}

\newsavebox{\backslashbox}
\sbox{\backslashbox}{\texttt{\symbol{92}}}
\newcommand{\bs}{\usebox{\backslashbox}}

\begin{document}

\logpage{[ 0, 0, 0 ]}
\begin{titlepage}
\begin{center}{\Huge \textbf{ \textsf{Repsn} }}\\[1cm]
\hypersetup{pdftitle= \textsf{Repsn} }
\markright{\scriptsize \mbox{}\hfill  \textsf{Repsn}  \hfill\mbox{}}
{\Large \textbf{  A \textsf{GAP}4 Package\\
 for constructing representations of finite groups ~ }}\\[1cm]
{Version 2.1}\\[1cm]
{February 2007}\\[1cm]
\mbox{}\\[2cm]
{\large \textbf{ Vahid Dabbaghian-Abdoly\\
  ~    }}\\
\hypersetup{pdfauthor= Vahid Dabbaghian-Abdoly\\
  ~    }
\end{center}\vfill

\mbox{}\\
{\mbox{}\\
\small \noindent \textbf{ Vahid Dabbaghian-Abdoly\\
  ~    } --- Email: \href{mailto:// vdabbagh@sfu.ca}{\texttt{ vdabbagh@sfu.ca}}\\
 --- Homepage: \href{http://www.sfu.ca/~vdabbagh}{\texttt{http://www.sfu.ca/\~{}vdabbagh}}\\
 --- Address: \begin{minipage}[t]{8cm}\noindent
 Department of Mathematics,\\
 Simon Fraser University,\\
 Burnaby, British Columbia,\\
 V5A 1S6 Canada. \end{minipage}
}\\
\end{titlepage}

\newpage\setcounter{page}{2}
{\small 
\section*{Copyright}
\logpage{[ 0, 0, 1 ]}
  {\copyright} 2004 Vahid Dabbaghian-Abdoly. }\\[1cm]
{\small 
\section*{Acknowledgements}
\logpage{[ 0, 0, 2 ]}
  

This package was obtained during my Ph.D. studies at Carleton University. I
would like to express deep gratitude to my supervisor Professor John D. Dixon
whose guidance and support were crucial for the successful completion of this
project. I also thank Professor Charles Wright and referees for pointing out
some important comments to improve \textsf{Repsn}. 

This documentation was prepared with the \textsf{GAPDoc} package by Frank L�beck and Max Neunh�ffer. }\\[1cm]
\newpage

\def\contentsname{Contents\logpage{[ 0, 0, 3 ]}}

\tableofcontents
\newpage

  
\chapter{\textcolor{Chapter }{Introduction}}\logpage{[ 1, 0, 0 ]}
{
 

This manual describes the \textsf{Repsn} package for computing matrix representations in characteristic zero of finite
groups. Most of the functions in \textsf{Repsn} have been written according to the algorithm described in the author's Ph.D
thesis \cite{Dab-03} (see \cite{Dab-05}). 

For constructing representations of simple groups and their covers we use the
algorithm described in \cite{Dix-93}. To use this algorithm for constructing a representation of a group $G$ affording an irreducible character $chi$ of $G$, we need to have a subgroup $H$ of $G$ such that the restriction of $chi$ to $H$ has a linear constituent with multiplicity one. In this case we say $H$ is a \emph{character subgroup} relative to $chi$ (or a $chi$-subgroup). A $chi$-subgroup for each irreducible character $chi$ of degree less than 32 of simple groups and their covers are listed in \cite{Dab-03}. 

All \textsf{Repsn} functions are written entirely in the \textsf{GAP} language. It is proved in \cite{Dab-05} that the algorithm is correct for any group with a character of degree less
than 32. Indeed, if the group is solvable, there is no restriction on the
character degree. In practice the program is quite fast when the degree is
small, but can be very slow when it is necessary to call one of the
subprograms which extend irreducible representations. In the latter case the
number of element wise operations required to extend a representation of
degree $d$ is proportional to $d^6$. 

\textsf{Repsn} is implemented in the \textsf{GAP} language, and runs on any system supporting \textsf{GAP}4. The \textsf{Repsn} package is loaded into the current \textsf{GAP} session with the command 
\begin{verbatim}   gap> LoadPackage( "repsn" ); 
\end{verbatim}
 (see section \emph{Loading a GAP Package} in the \textsf{GAP} Reference Manual). One could install the \textsf{Repsn} package on \textsf{GAP}4.3. In this case it is loaded with the command 
\begin{verbatim}   gap> RequirePackage( "repsn" ); 
\end{verbatim}
 

 \textsf{Repsn} has been developed by\\
 Vahid Dabbaghian-Abdoly \\
 Department of Mathematics\\
 Simon Fraser University\\
 Burnaby, British Columbia,\\
 V5A 1S6 Canada.\\
 e-mail: vdabbagh@sfu.ca\\
 

 Please send bug reports, suggestions and other comments to this e-mail
address. }

 
\chapter{\textcolor{Chapter }{Irreducible Representations}}\logpage{[ 2, 0, 0 ]}
{
 Let $G$ be a finite group and $chi$ be an ordinary irreducible character of $G$. In this chapter we introduce some functions to construct a complex
representation $R$ of $G$ affording $chi$. We proceed recursively, reducing the problem to smaller subgroups of $G$ or characters of smaller degree until we obtain a problem which we can deal
with directly. Inputs of most of the functions are a given group $G$, and an irreducible character $chi$. The output is a mapping (representation) which assigns to each generator $x$ of $G$ a matrix $R(x)$. We can use these functions for all groups and all irreducible characters $chi$ of degree less than 32 although in principle the same methods can be extended
to characters of larger degree. The main methods in these functions which are
used to construct representations of finite groups are induction, extension,
tensor product and Dixon's method (for constructing representations of simple
groups and their covers). 
\section{\textcolor{Chapter }{Constructing Representations}}\logpage{[ 2, 1, 0 ]}
{
  This section introduces the main function to compute a representation of a
finite group $G$ affording an irreducible character $chi$ of $G$. 

\subsection{\textcolor{Chapter }{IrreducibleAffordingRepresentation}}
\logpage{[ 2, 1, 1 ]}\nobreak
{\noindent\textcolor{FuncColor}{$\Diamond$\ \texttt{IrreducibleAffordingRepresentation( chi )\index{IrreducibleAffordingRepresentation@\texttt{IrreducibleAffordingRepresentation}}
\label{IrreducibleAffordingRepresentation}
}\hfill{\scriptsize (function)}}\\


 called with an irreducible character \mbox{\texttt{chi}} of a group $G$, this function returns a mapping (representation) which maps each generator
of $G$ to a $d*d$ matrix, where $d$ is the degree of \mbox{\texttt{chi}}. The group generated by these matrices (the image of the map) is a matrix
group which is isomorphic to $G$ modulo the kernel of the map. If $G$ is a solvable group then there is no restriction on the degree of \mbox{\texttt{chi}}. In the case that $G$ is not solvable and the character \mbox{\texttt{chi}} has degree bigger than 31 the output maybe is not correct. In this case
sometimes the output mapping does not afford the given character or it does
not return any mapping. }

 

\subsection{\textcolor{Chapter }{IsAffordingRepresentation}}
\logpage{[ 2, 1, 2 ]}\nobreak
{\noindent\textcolor{FuncColor}{$\Diamond$\ \texttt{IsAffordingRepresentation( chi, rep )\index{IsAffordingRepresentation@\texttt{IsAffordingRepresentation}}
\label{IsAffordingRepresentation}
}\hfill{\scriptsize (function)}}\\


 If \mbox{\texttt{chi}} and \mbox{\texttt{rep}} are a character and a representation of a group $G$, respectively, then \texttt{IsAffordingRepresentation} returns \texttt{true} if the trace of \mbox{\texttt{rep(x)}} equals \mbox{\texttt{chi(x)}} for all elements $x$ in $G$. 
\begin{Verbatim}[fontsize=\small,frame=single,label=Example]
  gap> G := GL(2,7);:
  gap> chi := Irr(G)[ 29 ];;
  gap> rep := IrreducibleAffordingRepresentation( chi );
  CompositionMapping( [(8,15,22,29,36,43)(9,16,23,30,37,44)
  (10,17,24,31,38,45)(11,18,25,32,39,46)(12,19,26,33,40,47)
  (13,20,27,34,41,48)(14,21,28,35,42,49), (2,29,12)(3,36,20)
  (4,43,28)(5,8,30)(6,15,38)(7,22,46)(9,44,14)(10,16,17)
  (11,37,27)(13,23,39)(18,24,25)(19,45,35)(21,31,47)
  (26,32,33)(34,40,41)(42,48,49) ] ->
  [ [ [ 0, 0, 0, -1, 0, 0, 0 ],
      [ 1, 0, -1, -1, 1, 0, -1 ] 
      [ 2, -1, -2, -2, 1, 2, -1 ],
      [ 0, 0, -1, 0, 0, 0, 0 ],
      [ 1, 0, -2, 0, 0, 1, -1 ],
      [ 1, 0, -2, -1, 1, 1, -1 ],
      [ -2, 1, 1, 1, -1, -1, 0 ] ],
    [ [ 1, -1, -1, -1, 0, 2, -1 ],
      [ 0, 0, 1, 0, 0, 0, 0 ],
      [ 0, 0, 0, 0, 0, 1, 0 ],
      [ 0, 1, -1, 0, 0, 0, -1 ],
      [ 0, 1, 0, 1, 0, -1, 0 ],
      [ 0, 1, 0, 0, 0, 0, 0 ],
      [ 0, 0, 0, 0, -1, 0, 0 ] ] ], (action isomorphism) )
  gap> IsAffordingRepresentation( chi, rep );
  true
   
\end{Verbatim}
 

We can obtain the size of the image of this representation by \texttt{Size(Image(rep))} and compute the value for an arbitrary element $x$ in $G$ by \texttt{x}{\textasciicircum}\texttt{rep}. }

 }

 
\section{\textcolor{Chapter }{Induction}}\logpage{[ 2, 2, 0 ]}
{
  

\subsection{\textcolor{Chapter }{InducedSubgroupRepresentation}}
\logpage{[ 2, 2, 1 ]}\nobreak
{\noindent\textcolor{FuncColor}{$\Diamond$\ \texttt{InducedSubgroupRepresentation( G, rep )\index{InducedSubgroupRepresentation@\texttt{InducedSubgroupRepresentation}}
\label{InducedSubgroupRepresentation}
}\hfill{\scriptsize (function)}}\\


 computes a representation of \mbox{\texttt{G}} induced from the representation \mbox{\texttt{rep}} of a subgroup $H$ of \mbox{\texttt{G}}. If \mbox{\texttt{rep}} has degree $d$ then the degree of the output representation is $d*|G:H|$. 
\begin{Verbatim}[fontsize=\small,frame=single,label=Example]
  gap> G := SymmetricGroup( 6 );;
  gap> H := AlternatingGroup( 6 );;
  gap> chi := Irr( H )[ 2 ];;
  gap> rep := IrreducibleAffordingRepresentation( chi );;
  gap> InducedSubgroupRepresentation( G, rep ); 
  [ (1,2,3,4,5,6), (1,2) ] ->
  [ [ [ 0, 0, 0, 0, 0, 1, 1, -1, -1, -1 ],
      [ 0, 0, 0, 0, 0, 1, 0, -1, 0, -1 ],
      [ 0, 0, 0, 0, 0, 1, 0, 0, -1, -1 ],
      [ 0, 0, 0, 0, 0, 1, 0, 0, 0, 0 ],
      [ 0, 0, 0, 0, 0, 0, 1, -1, 0, -1 ],
      [ 1, 1, -1, -1, -1, 0, 0, 0, 0, 0 ],
      [ 1, 0, 0, -1, -1, 0, 0, 0, 0, 0 ],
      [ 1, 0, 0, 0, 0, 0, 0, 0, 0, 0 ],
      [ 1, 0, -1, 0, -1, 0, 0, 0, 0, 0 ],
      [ 0, 1, 0, -1, -1, 0, 0, 0, 0, 0 ] ],
    [ [ 0, 0, 0, 0, 0, 1, 0, 0, 0, 0 ],
      [ 0, 0, 0, 0, 0, 0, 1, 0, 0, 0 ],
      [ 0, 0, 0, 0, 0, 0, 0, 0, 1, 0 ],
      [ 0, 0, 0, 0, 0, 0, 0, 1, 0, 0 ],
      [ 0, 0, 0, 0, 0, 1, 1, -1, -1, -1 ],
      [ 1, 0, 0, 0, 0, 0, 0, 0, 0, 0 ],
      [ 0, 1, 0, 0, 0, 0, 0, 0, 0, 0 ],
      [ 0, 0, 0, 1, 0, 0, 0, 0, 0, 0 ],
      [ 0, 0, 1, 0, 0, 0, 0, 0, 0, 0 ],
      [ 1, 1, -1, -1, -1, 0, 0, 0, 0, 0 ] ] ]
  	
\end{Verbatim}
 }

 }

 
\section{\textcolor{Chapter }{Extension}}\logpage{[ 2, 3, 0 ]}
{
  In this section we introduce some functions for extending a representation of
a subgroup to the whole group. 

\subsection{\textcolor{Chapter }{ExtendedRepresentation}}
\logpage{[ 2, 3, 1 ]}\nobreak
{\noindent\textcolor{FuncColor}{$\Diamond$\ \texttt{ExtendedRepresentation( chi, rep )\index{ExtendedRepresentation@\texttt{ExtendedRepresentation}}
\label{ExtendedRepresentation}
}\hfill{\scriptsize (function)}}\\


 Suppose $H$ is a subgroup of a group $G$ and \mbox{\texttt{chi}} is an irreducible character of $G$ such that the restriction of \mbox{\texttt{chi}} to $H$, $phi$ say, is irreducible. If \mbox{\texttt{rep}} is an irreducible representation of $H$ affording $phi$ then \texttt{ExtendedRepresentation} extends the representation \mbox{\texttt{rep}} of $H$ to a representation of $G$ affording \mbox{\texttt{chi}}. This function call can be quite expensive when the representation \mbox{\texttt{rep}} has a large degree. 
\begin{Verbatim}[fontsize=\small,frame=single,label=Example]
  gap> G := AlternatingGroup( 6 );;
  gap> H := Group([ (1,2,3,4,6), (1,4)(5,6) ]);;
  gap> chi := Irr( G )[ 2 ];;
  gap> phi := RestrictedClassFunction( chi, H );;
  gap> IsIrreducibleCharacter( phi );
  true
  gap> rep := IrreducibleAffordingRepresentation( phi );;
  gap> ext := ExtendedRepresentation( chi, rep );
  #I  Need to extend a representation of degree 5. This may take a while.
  [ (1,2,3,4,5), (4,5,6) ] -> [
  [ [ 0, 1, 0, -1, -1 ],
    [ 0, 0, 0, 1, 0 ],
    [ -1, -1, -1, 0, 0 ],
    [ 0, 0, 0, 0, -1 ],
    [ 0, 0, 1, 1, 1 ] ],
  [ [ 1, 0, 1, 0, 1 ],
    [ 0, 1, 0, 0, 0 ],
    [ -1, -1, 0, 1, 0 ],
    [ 1, 1, 1, 0, 0 ],
    [ 0, 0, -1, 0, 0 ] ] ]
  gap> IsAffordingRepresentation( chi, ext );
  true
  		    
\end{Verbatim}
 }

 

\subsection{\textcolor{Chapter }{ExtendedRepresentationNormal}}
\logpage{[ 2, 3, 2 ]}\nobreak
{\noindent\textcolor{FuncColor}{$\Diamond$\ \texttt{ExtendedRepresentationNormal( chi, rep )\index{ExtendedRepresentationNormal@\texttt{ExtendedRepresentationNormal}}
\label{ExtendedRepresentationNormal}
}\hfill{\scriptsize (function)}}\\


 Suppose $H$ is a normal subgroup of a group $G$ and \mbox{\texttt{chi}} is an irreducible character of $G$ such that the restriction of \mbox{\texttt{chi}} to $H$, $phi$ say, is irreducible. If \mbox{\texttt{rep}} is an irreducible representation of $H$ affording $phi$ then \texttt{ExtendedRepresentationNormal} extends the representation \mbox{\texttt{rep}} of $H$ to a representation of $G$ affording \mbox{\texttt{chi}}. This function is more efficient than \texttt{ExtendedRepresentation}. 
\begin{Verbatim}[fontsize=\small,frame=single,label=Example]
  gap> G := GL(2,7);;
  gap> chi := Irr( G )[ 29 ];;
  gap> H := SL(2,7);;
  gap> phi := RestrictedClassFunction( chi, H );;
  gap> IsIrreducibleCharacter( phi );
  true
  gap> rep := IrreducibleAffordingRepresentation( phi );;
  gap> ext := ExtendedRepresentationNormal( chi, rep );
  #I  Need to extend a representation of degree 7. This may take a while.
  CompositionMapping( [(8,15,22,29,36,43)(9,16,23,30,37,44)
   (10,17,24,31,38,45)(11,18,25,32,39,46)(12,19,26,33,40,47)
   (13,20,27,34,41,48)(14,21,28,35,42,49),(2,29,12)(3,36,20)
   (4,43,28)(5,8,30)(6,15,38)(7,22,46)(9,44,14)(10,16,17)
   (11,37,27)(13,23,39)(18,24,25)(19,45,35)(21,31,47)
   (26,32,33)(34,40,41)(42,48,49) ] ->
  [ [ [ -1, 0, 0, 1, 0, -1, 0 ], [ -1, 0, 0, 0, 0, 0, 0 ],
    [ -1, 1, 0, 0, -1, 0, 0 ], [ 0, -1, 0, 0, 0, 0, 0 ],
    [ -1, -1, 1, 0, 1, -1, 0 ], [ 0, 0, 0, -1, 0, 0, 0 ],
    [ -1, 0, 1, -1, 1, 0, -1 ] ],
    [ [ 1, -1, 0, 1, 0, -1, 1 ], [ 1, 0, -1, 1, -1, 0, 1 ],
    [ 1, -1, 0, 1, -1, 0, 1 ], [ 0, 0, -1, 0, 0, 0, 0 ],
    [ -1, 0, 0, 1, 0, -1, 0 ], [ -1, 0, 0, 0, 0, 0, 0 ],
    [ -1, 1, 0, 0, -1, 0, 0 ] ] ], (action isomorphism) )	
  gap> IsAffordingRepresentation( chi, ext );
  true
  	  
\end{Verbatim}
 }

 }

 
\section{\textcolor{Chapter }{Character Subgroups}}\logpage{[ 2, 4, 0 ]}
{
  If $chi$ is an irreducible character of a group $G$ and $H$ is a subgroup of $G$ such that the restriction of $chi$ to $H$ has a linear constituent with multiplicity one, then we call $H$ a character subgroup relative to $chi$ or a $chi$-subgroup. 

\subsection{\textcolor{Chapter }{CharacterSubgroupRepresentation}}
\logpage{[ 2, 4, 1 ]}\nobreak
{\noindent\textcolor{FuncColor}{$\Diamond$\ \texttt{CharacterSubgroupRepresentation( chi )\index{CharacterSubgroupRepresentation@\texttt{CharacterSubgroupRepresentation}}
\label{CharacterSubgroupRepresentation}
}\hfill{\scriptsize (function)}}\\
\noindent\textcolor{FuncColor}{$\Diamond$\ \texttt{CharacterSubgroupRepresentation( chi, H )\index{CharacterSubgroupRepresentation@\texttt{CharacterSubgroupRepresentation}}
\label{CharacterSubgroupRepresentation}
}\hfill{\scriptsize (function)}}\\


 returns a representation affording \mbox{\texttt{chi}} by finding a \mbox{\texttt{chi}}-subgroup and using the method described in \cite{Dix-93}. If the second argument is a \mbox{\texttt{chi}}-subgroup then it returns a representation affording \mbox{\texttt{chi}} without searching for a \mbox{\texttt{chi}}-subgroup. In this case an error is signalled if no \mbox{\texttt{chi}}-subgroup exists. }

 

\subsection{\textcolor{Chapter }{IsCharacterSubgroup}}
\logpage{[ 2, 4, 2 ]}\nobreak
{\noindent\textcolor{FuncColor}{$\Diamond$\ \texttt{IsCharacterSubgroup( chi, H )\index{IsCharacterSubgroup@\texttt{IsCharacterSubgroup}}
\label{IsCharacterSubgroup}
}\hfill{\scriptsize (function)}}\\


 is \texttt{true} if \mbox{\texttt{H}} is a \mbox{\texttt{chi}}-subgroup and \texttt{false} otherwise. 
\begin{Verbatim}[fontsize=\small,frame=single,label=Example]
  gap> G := AlternatingGroup( 8 );;
  gap> chi := Irr( G )[ 2 ];;
  gap> H := AlternatingGroup( 3 );;
  gap> IsCharacterSubgroup( chi, H );
  true
  gap> rep := CharacterSubgroupRepresentation( G, chi, H );
  [ (1,2,3,4,5,6,7), (6,7,8) ] -> [ [ [
   1/3*E(3)+2/3*E(3)^2, 0, 0, -E(3), 0, -1/3*E(3)-2/3*E(3)^2, 1 ],
     [ 2/3*E(3)+4/3*E(3)^2, 0, 1, 0, 0, 1/3*E(3)-1/3*E(3)^2, 0 ],
     [ 2/3*E(3)+4/3*E(3)^2, 0, 0, 1, 0, 1/3*E(3)-1/3*E(3)^2, 0 ],
     [ E(3)^2, 0, 0, 0, 0, 0, 0 ],
     [ 2/3*E(3)+4/3*E(3)^2, 0, 0, 0, 1, 1/3*E(3)-1/3*E(3)^2, 0 ],
     [ -2/3*E(3)-1/3*E(3)^2, 0, 0, -1, 0, 2/3*E(3)+1/3*E(3)^2, E(3)^2 ],
     [ 0, 1, 0, 0, 0, 0, 0 ] ],
   [ [ 1, 0, 0, 0, 0, 0, 0 ], [ 0, 1, 0, 0, 0, 0, 0 ],
     [ 0, 0, 0, 1, 0, 0, 0 ], [ 0, 0, 0, 0, 1, 0, 0 ],
     [ 0, 0, 1, 0, 0, 0, 0 ], [ 0, 0, 0, 0, 0, 1, 0 ],
     [ 0, 0, 0, -E(3), E(3), 0, 1 ] ] ]
  	
\end{Verbatim}
 }

 

\subsection{\textcolor{Chapter }{AllCharacterPSubgroups}}
\logpage{[ 2, 4, 3 ]}\nobreak
{\noindent\textcolor{FuncColor}{$\Diamond$\ \texttt{AllCharacterPSubgroups( G, chi )\index{AllCharacterPSubgroups@\texttt{AllCharacterPSubgroups}}
\label{AllCharacterPSubgroups}
}\hfill{\scriptsize (function)}}\\


 returns a list of all $p$-subgroups of \mbox{\texttt{G}} which are \mbox{\texttt{chi}}-subgroups. The subgroups are chosen up to conjugacy in \mbox{\texttt{G}}. }

 

\subsection{\textcolor{Chapter }{AllCharacterStandardSubgroups}}
\logpage{[ 2, 4, 4 ]}\nobreak
{\noindent\textcolor{FuncColor}{$\Diamond$\ \texttt{AllCharacterStandardSubgroups( G, chi )\index{AllCharacterStandardSubgroups@\texttt{AllCharacterStandardSubgroups}}
\label{AllCharacterStandardSubgroups}
}\hfill{\scriptsize (function)}}\\


 returns a list containing well described subgroups of \mbox{\texttt{G}} which are \mbox{\texttt{chi}}-subgroups. This list may contain Sylow subgroups and their derived subgroups,
normalizers and centralzers in \mbox{\texttt{G}}. }

 

\subsection{\textcolor{Chapter }{AllCharacterSubgroups}}
\logpage{[ 2, 4, 5 ]}\nobreak
{\noindent\textcolor{FuncColor}{$\Diamond$\ \texttt{AllCharacterSubgroups( G, chi )\index{AllCharacterSubgroups@\texttt{AllCharacterSubgroups}}
\label{AllCharacterSubgroups}
}\hfill{\scriptsize (function)}}\\


 returns a list of all \mbox{\texttt{chi}}-subgroups of \mbox{\texttt{G}} among the lattice of subgroups. This function call can be quite expensive for
larger groups. The call is expensive in particular if the lattice of subgroups
of the given group is not yet known. }

 }

 
\section{\textcolor{Chapter }{Equivalent Representation}}\logpage{[ 2, 5, 0 ]}
{
 

\subsection{\textcolor{Chapter }{EquivalentRepresentation}}
\logpage{[ 2, 5, 1 ]}\nobreak
{\noindent\textcolor{FuncColor}{$\Diamond$\ \texttt{EquivalentRepresentation( rep )\index{EquivalentRepresentation@\texttt{EquivalentRepresentation}}
\label{EquivalentRepresentation}
}\hfill{\scriptsize (function)}}\\


 computes an equivalent representation to an irreducible representation \mbox{\texttt{rep}} by transforming \mbox{\texttt{rep}} to a new basis by spinning up one vector (i.e. getting the other basis vectors
as images under the first one under words in the generators). If the input
representation, \mbox{\texttt{rep}}, is reducible then \texttt{EquivalentRepresentation} does not return any mapping. In this case see section 3. 
\begin{Verbatim}[fontsize=\small,frame=single,label=Example]
  gap> G := SymmetricGroup( 7 );;
  gap> chi := Irr( G )[ 2 ];;
  gap> rep := CharacterSubgroupRepresentation( G, chi );;
  gap> equ := EquivalentRepresentation( rep );
  [ (1,2,3,4,5,6,7), (1,2) ] ->
  [ [ [ 0, 0, 0, E(5)+E(5)^2+E(5)^3+2*E(5)^4, -1, -E(5)-E(5)^2-E(5)^3-2*E(5)^4 ],
     [ E(5)^3-E(5)^4, E(5)^2+E(5)^3+E(5)^4, E(5)+E(5)^3-E(5)^4, -E(5)+E(5)^2
            -3*E(5)^3-E(5)^4, -E(5)-E(5)^3+E(5)^4, 2*E(5)-2*E(5)^2+2*E(5)^3 ]
      , [ 0, 0, 0, 1, 0, 0 ],
     [ 0, 4/5*E(5)+3/5*E(5)^2+2/5*E(5)^3+1/5*E(5)^4, E(5), 1, -E(5),
         6/5*E(5)+2/5*E(5)^2+3/5*E(5)^3+4/5*E(5)^4 ], [ 0, 1, 0, 0, 0, 0 ],
     [ 0, 0, E(5), 1, -E(5), 2*E(5)+E(5)^2+E(5)^3+E(5)^4 ] ],
   [ [ -1, 0, E(5)+E(5)^2+E(5)^3+2*E(5)^4, -E(5)-E(5)^2-3*E(5)^4,
      -E(5)-E(5)^2-E(5)^3-2*E(5)^4, E(5)+E(5)^2+3*E(5)^4 ],
    [ 0, -1, 0, 0, 0, 0 ],
    [ 0, 0, 0, E(5)+E(5)^2+E(5)^3+2*E(5)^4, -1, -E(5)-E(5)^2-E(5)^3-2*E(5)^4
       ], [ 0, 0, -1, -E(5)^4, 1, E(5)+E(5)^2+E(5)^3+2*E(5)^4 ],
    [ 0, 0, -E(5)^4, -E(5)^3+E(5)^4, E(5)+E(5)^2+E(5)^3+2*E(5)^4,
        E(5)^3-E(5)^4 ], [ 0, 0, 0, 0, 0, -1 ] ] ]
  gap> IsAffordingRepresentation( chi, equ );
  true
  	
\end{Verbatim}
 }

 }

 }

 
\chapter{\textcolor{Chapter }{Reducible Representations}}\logpage{[ 3, 0, 0 ]}
{
 In this chapter we introduce some functions which deal with a complex
reducible representation $R$ of a finite group $G$. 
\section{\textcolor{Chapter }{Constituents of Representations}}\logpage{[ 3, 1, 0 ]}
{
  

\subsection{\textcolor{Chapter }{ConstituentsOfRepresentation}}
\logpage{[ 3, 1, 1 ]}\nobreak
{\noindent\textcolor{FuncColor}{$\Diamond$\ \texttt{ConstituentsOfRepresentation( rep )\index{ConstituentsOfRepresentation@\texttt{ConstituentsOfRepresentation}}
\label{ConstituentsOfRepresentation}
}\hfill{\scriptsize (function)}}\\


 called with a representation \mbox{\texttt{rep}} of a group $G$. This function returns a list of irreducible representations of $G$ which are constituents of \mbox{\texttt{rep}}, and their corresponding multiplicities. For example, if \mbox{\texttt{rep}} is a representation of $G$ affording a character $X$ such that $X = mY + nZ$, where $Y$ and $Z$ are irreducible characters of $G$, and $m$ and $n$ are the corresponding multiplicities, then \texttt{ConstituentsOfRepresentation} returns $[[m, S]$, $[n, T]]$ where $S$ and $T$ are irreducible representations of $G$ affording $Y$ and $Z$, respectively. This function call can be quite expensive when $G$ is a large group. }

 

\subsection{\textcolor{Chapter }{IsReducibleRepresentation}}
\logpage{[ 3, 1, 2 ]}\nobreak
{\noindent\textcolor{FuncColor}{$\Diamond$\ \texttt{IsReducibleRepresentation( rep )\index{IsReducibleRepresentation@\texttt{IsReducibleRepresentation}}
\label{IsReducibleRepresentation}
}\hfill{\scriptsize (function)}}\\


 If \mbox{\texttt{rep}} is a representation of a group $G$ then \texttt{IsReducibleRepresentation} returns \texttt{true} if \mbox{\texttt{rep}} is a reducible representation of $G$. }

 }

 
\section{\textcolor{Chapter }{Block Representations}}\logpage{[ 3, 2, 0 ]}
{
  

\subsection{\textcolor{Chapter }{EquivalentBlockRepresentation}}
\logpage{[ 3, 2, 1 ]}\nobreak
{\noindent\textcolor{FuncColor}{$\Diamond$\ \texttt{EquivalentBlockRepresentation( rep )\index{EquivalentBlockRepresentation@\texttt{EquivalentBlockRepresentation}}
\label{EquivalentBlockRepresentation}
}\hfill{\scriptsize (function)}}\\
\noindent\textcolor{FuncColor}{$\Diamond$\ \texttt{EquivalentBlockRepresentation( list )\index{EquivalentBlockRepresentation@\texttt{EquivalentBlockRepresentation}}
\label{EquivalentBlockRepresentation}
}\hfill{\scriptsize (function)}}\\


 If \mbox{\texttt{rep}} is a reducible representation of a group $G$, this function returns a block diagonal representation of $G$ equivalent to \mbox{\texttt{rep}}. If \mbox{\texttt{ list }} $= [[m1, R1]$, $[m2, R2]$, ... , $[mt, Rt]]$ is a list of irreducible representations $R1$, $R2$, ... , $Rt$ of $G$ with multiplicities $m1$, $m2$, ... , $mt$, then \texttt{EquivalentBlockRepresentation} returns a block diagonal representation of $G$ containing the blocks $R1$, $R2$, ... , $Rt$. 
\begin{Verbatim}[fontsize=\small,frame=single,label=Example]
  gap> G := AlternatingGroup( 5 );;
  gap> H := SylowSubgroup( G, 2 );;
  gap> chi := TrivialCharacter( H );;
  gap> Hrep := IrreducibleAffordingRepresentation( chi );;
  gap> rep := InducedSubgroupRepresentation( G, Hrep );;
  gap> IsReducibleRepresentation( rep );
  true
  gap> con := ConstituentsOfRepresentation( rep );
  [ [ 1, [ (1,2,3,4,5), (3,4,5) ] -> [ [ [ 1 ] ], [ [ 1 ] ] ] ],
    [ 1, [ (1,2,3,4,5), (3,4,5) ] ->
          [ [ [ E(3), -1/3*E(3)-2/3*E(3)^2, 0, 1/3*E(3)-1/3*E(3)^2 ],
              [ 1, -4/3*E(3)+1/3*E(3)^2, E(3), -2/3*E(3)-1/3*E(3)^2 ],
              [ 1, -E(3), E(3), 0 ],
              [ 1, -1/3*E(3)+1/3*E(3)^2, 1, 1/3*E(3)+2/3*E(3)^2 ] ],
            [ [ 1, -2/3*E(3)-1/3*E(3)^2, 0, 2/3*E(3)+1/3*E(3)^2 ],
              [ 0, -E(3), E(3), 1 ],
              [ 0, -4/3*E(3)-2/3*E(3)^2, E(3), -2/3*E(3)-1/3*E(3)^2 ],
              [ 0, 0, 1, 0 ] ] ] ],
    [ 2, [ (1,2,3,4,5), (3,4,5) ] -> 
          [ [ [ -1, 1, 1, 1, -1 ], 
              [ 0, 0, 0, 0, 1 ],
              [ -1, 0, 0, 1, -1 ],
              [ 0, 0, 1, 0, 0 ], 
              [ 0, -1, 0, -1, 1 ] ],
            [ [ 0, 0, 0, 0, 1 ],
              [ 0, -1, -1, -1, 0 ],
              [ 0, 1, 0, 0, 0 ],
              [ 0, 0, 0, 1, 0 ],
              [ -1, 0, 0, 1, -1 ] ] ] ] ]
  gap> EquivalentBlockRepresentation( con );
  [ (1,2,3,4,5), (3,4,5) ] ->
  [ [ [ 1, 0, 0, 0, 0, 0, 0, 0, 0, 0, 0, 0, 0, 0, 0 ],
      [ 0, E(3), -1/3*E(3)-2/3*E(3)^2, 0, 1/3*E(3)-1/3*E(3)^2, 0, 
        0, 0, 0, 0,  0, 0, 0, 0, 0 ],
      [ 0, 1, -4/3*E(3)+1/3*E(3)^2, E(3), -2/3*E(3)-1/3*E(3)^2, 0, 
        0, 0, 0, 0, 0, 0, 0, 0, 0 ],
      [ 0, 1, -E(3), E(3), 0, 0, 0, 0, 0, 0, 0, 0, 0, 0, 0 ],
      [ 0, 1, -1/3*E(3)+1/3*E(3)^2, 1, 1/3*E(3)+2/3*E(3)^2, 0, 0, 
        0, 0, 0, 0, 0, 0, 0, 0 ], 
      [ 0, 0, 0, 0, 0, -1, 1, 1, 1, -1, 0, 0, 0, 0, 0 ],
      [ 0, 0, 0, 0, 0, 0, 0, 0, 0, 1, 0, 0, 0, 0, 0 ],
      [ 0, 0, 0, 0, 0, -1, 0, 0, 1, -1, 0, 0, 0, 0, 0 ],
      [ 0, 0, 0, 0, 0, 0, 0, 1, 0, 0, 0, 0, 0, 0, 0 ],
      [ 0, 0, 0, 0, 0, 0, -1, 0, -1, 1, 0, 0, 0, 0, 0 ],
      [ 0, 0, 0, 0, 0, 0, 0, 0, 0, 0, -1, 1, 1, 1, -1 ],
      [ 0, 0, 0, 0, 0, 0, 0, 0, 0, 0, 0, 0, 0, 0, 1 ],
      [ 0, 0, 0, 0, 0, 0, 0, 0, 0, 0, -1, 0, 0, 1, -1 ],
      [ 0, 0, 0, 0, 0, 0, 0, 0, 0, 0, 0, 0, 1, 0, 0 ],
      [ 0, 0, 0, 0, 0, 0, 0, 0, 0, 0, 0, -1, 0, -1, 1 ] ],
    [ [ 1, 0, 0, 0, 0, 0, 0, 0, 0, 0, 0, 0, 0, 0, 0 ],
      [ 0, 1, -2/3*E(3)-1/3*E(3)^2, 0, 2/3*E(3)+1/3*E(3)^2, 0, 0, 
        0, 0, 0, 0, 0, 0, 0, 0 ],
      [ 0, 0, -E(3), E(3), 1, 0, 0, 0, 0, 0, 0, 0, 0, 0, 0 ],
      [ 0, 0, -4/3*E(3)-2/3*E(3)^2, E(3), -2/3*E(3)-1/3*E(3)^2, 0, 
        0, 0, 0, 0, 0, 0, 0, 0, 0 ],
      [ 0, 0, 0, 1, 0, 0, 0, 0, 0, 0, 0, 0, 0, 0, 0 ],
      [ 0, 0, 0, 0, 0, 0, 0, 0, 0, 1, 0, 0, 0, 0, 0 ],
      [ 0, 0, 0, 0, 0, 0, -1, -1, -1, 0, 0, 0, 0, 0, 0 ],
      [ 0, 0, 0, 0, 0, 0, 1, 0, 0, 0, 0, 0, 0, 0, 0 ],
      [ 0, 0, 0, 0, 0, 0, 0, 0, 1, 0, 0, 0, 0, 0, 0 ],
      [ 0, 0, 0, 0, 0, -1, 0, 0, 1, -1, 0, 0, 0, 0, 0 ],
      [ 0, 0, 0, 0, 0, 0, 0, 0, 0, 0, 0, 0, 0, 0, 1 ],
      [ 0, 0, 0, 0, 0, 0, 0, 0, 0, 0, 0, -1, -1, -1, 0 ],
      [ 0, 0, 0, 0, 0, 0, 0, 0, 0, 0, 0, 1, 0, 0, 0 ],
      [ 0, 0, 0, 0, 0, 0, 0, 0, 0, 0, 0, 0, 0, 1, 0 ],
      [ 0, 0, 0, 0, 0, 0, 0, 0, 0, 0, -1, 0, 0, 1, -1 ] ] ]
   
\end{Verbatim}
 }

 }

 }

 \def\bibname{References\logpage{[ "Bib", 0, 0 ]}}

\bibliographystyle{alpha}
\bibliography{repsn}

\def\indexname{Index\logpage{[ "Ind", 0, 0 ]}}


\printindex

\newpage
\immediate\write\pagenrlog{["End"], \arabic{page}];}
\immediate\closeout\pagenrlog
\end{document}
